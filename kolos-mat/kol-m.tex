%Paweł Renc
\documentclass[a4paper, 11pt]{article}
\usepackage{amssymb}
\usepackage[polish]{babel}
\usepackage[utf8]{inputenc}
\usepackage[T1]{fontenc}
\usepackage{graphicx}
\usepackage{anysize}
\usepackage{amsthm}
%\usepackage{enumerate}
%\usepackage{times}
\usepackage{color}
\usepackage{multirow}

\usepackage{fancyhdr}
\fancyhead{}
\fancyfoot{}
\fancyhead[L]{\slshape LITERATURA}
\fancyfoot[R]{Strona \thepage}
\renewcommand{\headrulewidth}{0.4pt}
\renewcommand{\footrulewidth}{0.4pt}


\setcounter{section}{35}
\setcounter{page}{151}
\setcounter{footnote}{7}
\setcounter{table}{14}

\marginsize{4cm}{2cm}{2cm}{2cm}


\begin{document}
%\bibliographystyle{abbrv}
\pagestyle{fancy}
\section{Tabela i literatura}
\label{title}

\subsection{Instruckja}

Należy napisać document klasy {\em article}, który będzie miał postać taką jak Pani/Pan trzyma w ręku. Dopuszczalne są odstępstwa od tej postac, tj. rozszerzenia oraz zwężenia do uproszczonej postaci (za "dodatkowe" i "ujemne" punkty). Informacje o tych modyfikacjach znajdują się w przypisach\footnote{Zalecane parametry dokumentu to {\em a4paper,11pt}, marginesy: górny i dolny 2cm, lewy 4cm, aprawy 2cm.}.

Ta strona zawiera rozdział (sekcje) 36, a w nim dwa podrozdziały(podsekcje)\footnote{Najlepiej byłoby nadać numer rozdziału ustawiając odpowiedni licznik. W przypadku innych obiektów -- podobnie, ale należy traktować to jako kosmetykę. Ważniejsze jest, aby w odwołaniach do obiektów używać przypisanych im etykiet.} i spis literatury. Na początku pliku źródłowego, w komentarzu proszę wpisać swoje imię i nazwisko. Wersję źródłową (.tex i .bib) i wynikową (.pdf) proszę wysłąć na adres $\mathtt{ptm@agh.edu.pl}$\footnote{Te dwa zdania należy napisać kolorem czerwonym}.

\subsection{Zadania}
\begin{enumerate}
\item {W tabeli (\ref{tab}) przedstawiono klasyfikację stałych w języku C.

\begin{table}[!h]
\caption{Stałe w języku C}
\label{tab}
\begin{center}
\begin{tabular}{|c|c|c|c|c|}

\cline{5-5}
\multicolumn{4}{l|}{}&Przykłady \\
\hline
\multicolumn{4}{|l|}{Deklarowane stałe} & const int size =128 \\
\hline
\multicolumn{4}{|l|}{Stałe preprocesora} & \#define SIZE 256 \\
\hline
\multirow{7}{*}{Literały} & \multicolumn{3}{|l|}{łańcuch znakowy} & "koniec linii . \textbackslash n" \\

 \cline{2-5}& \multicolumn{2}{c|}{\multirow{2}{*}{znakowe}} & escape sequence & '\textbackslash n', '\textbackslash xa4'\\
\cline{4-5} &&& znak & 'A', '!'\\  

\cline{2-5}&\multirow{4}{*}{liczbowe}&\multirow{3}{*}{całkowite}& dziesiętne & 8743 \\
\cline{4-5}&&&ósemkowe&07464 \\
\cline{4-5}&&&szesnastkowe&0x5AFF \\
\cline{3-5}&&\multicolumn{2}{|c|}{zmiennoprzecinkowe}& 140.58\\

\hline
\end{tabular}
\end{center}
\end{table}

}
\item {Należy\footnote{Możliwe uproszczenia: znaki przy wypunktowaniach w każdej pozycji można zastąpić standardowym znakiem dla {\em"itemize".}}:

\begin{itemize}
\item[$\lozenge$] {znaleźć w sieci notki bibliograficzne dwóch pozycji\footnote{Z dokładnością do autrów i tytułów -- pozostałe dane mogą wskazywać na inne wydanie.},
}
\item[$\lozenge$] {utworzyć plik $\mathtt{.bib}$,}
\item[$\star$] {
odwołać się do znalezionych pozycji w literaturze \cite{kahneman2011thinking} lub \cite{kernighan2006c}.
}
\end{itemize}
}
\end{enumerate}

%\bibliography{bib}

\end{document}
